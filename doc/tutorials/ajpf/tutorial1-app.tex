\section{Implementation of BDI Modalities in \gwendolen}
\label{s:impl_bdi}

In \gwendolen\ the BDI modalities of the \ajpf\ property specification language are implemented as follows.

\begin{itemize}
\item $\lbelief{\AILagent}{f}$.  An agent, $\AILagent$, believes the formula, $f$, if $f$ appears in its belief base or is deducible from its goal base using its reasoning rules.
\item $\lgoal{\AILagent}{f}$.  An agent, $\AILagent$, has a goal $f$, if $f$ is a goal that appears in the agent's goal base.
\item $\lintention{\AILagent}{f}$.  An agent, $\AILagent$, has an intention $f$, if $f$ is a goal in the goal base a plan has been selected to achieve or perform the goal.
\item $\lintendtodo{\AILagent}{f}$.  An agent, $\AILagent$, intends to do $f$, if $f$ is an action that appears in the deed stack of some intention.
\end{itemize}

\subsection{Intending to Send a Message}
\gwendolen\ uses a special syntax for send actions (\texttt{.send(ag, :tell, c)}) which is not recognised by the property specification language.  If you want to check that a \gwendolen\ agent intends to send a messsage then you need to use the syntax \texttt{send(agname, number, c)} where \texttt{agname} is the name of the recipient, \texttt{number} is
\begin{description}
\item[1] For \texttt{:tell},
\item[2] For \texttt{:perform},
\item[3] For \texttt{:achieve}
\end{description}
and \texttt{c} is the content of the message.

